% =============================================================================
% Lecture01_Intro.tex -- Introducción al Desarrollo Económico
% Economía del Desarrollo | Universidad de Santiago de Chile (USACH)
%
% Compile from Slides/ with:
%   TEXINPUTS=../Preambles:$TEXINPUTS xelatex -interaction=nonstopmode Lecture01_Intro.tex
%   BIBINPUTS=..:$BIBINPUTS bibtex Lecture01_Intro
%   TEXINPUTS=../Preambles:$TEXINPUTS xelatex -interaction=nonstopmode Lecture01_Intro.tex
%   TEXINPUTS=../Preambles:$TEXINPUTS xelatex -interaction=nonstopmode Lecture01_Intro.tex
%
% Figures generated by: scripts/R/lecture01_figures.R
% =============================================================================

\documentclass[aspectratio=169, 12pt]{beamer}
% =============================================================================
% header.tex -- Beamer preamble for Economía del Desarrollo
% Universidad de Santiago de Chile (USACH)
%
% Usage: compile from Slides/ with TEXINPUTS=../Preambles:$TEXINPUTS
% Compiler: XeLaTeX only (fontspec requires it)
% =============================================================================

% --- Language ---
\usepackage[spanish,es-nodecimaldot]{babel}

% --- XeLaTeX Font Configuration ---
\usepackage{fontspec}
\setmainfont{Source Sans 3}
\setsansfont{Source Sans 3}

% --- Standard Math & Graphics ---
\usepackage{graphicx}
\usepackage{amsmath}
\usepackage{amssymb}
\usepackage{booktabs}
\usepackage{array}

% --- TikZ ---
\usepackage{tikz}
\usetikzlibrary{arrows.meta, positioning, shapes, calc, decorations.pathreplacing}
\usepackage{pgfplots}
\pgfplotsset{compat=1.18}

% --- Bibliography ---
\usepackage{natbib}
\bibliographystyle{apalike}

% --- Color Palette ---
\usepackage{xcolor}
\definecolor{primarynavy}{HTML}{1B2A4A}    % Deep Navy   -- headings, primary
\definecolor{warmteal}{HTML}{2A7F8E}       % Warm Teal   -- emphasis
\definecolor{accentamber}{HTML}{D4A843}    % Amber       -- highlights/alerts
\definecolor{lightgray}{HTML}{EDF1F5}      % Cool Gray   -- backgrounds
\definecolor{positivegreen}{HTML}{15803d}  % Forest Green -- good outcomes
\definecolor{negativered}{HTML}{b91c1c}    % Crimson     -- bad outcomes
\definecolor{slategray}{HTML}{525252}      % Slate Gray  -- muted text

% --- tcolorbox (required for custom environments below) ---
\usepackage[most]{tcolorbox}

% --- Hyperlinks ---
\usepackage{hyperref}
\hypersetup{
  colorlinks=true,
  linkcolor=primarynavy,
  citecolor=warmteal,
  urlcolor=warmteal
}

% =============================================================================
% CUSTOM COMMANDS
% =============================================================================

% \key{text}      -- bold teal, for key terms
% \muted{text}    -- slate gray, for caveats/context
% \positive{text} -- forest green bold, for good outcomes
% \negative{text} -- crimson bold, for bad outcomes/poverty

\newcommand{\key}[1]{\textbf{\textcolor{warmteal}{#1}}}
\newcommand{\muted}[1]{\textcolor{slategray}{#1}}
\newcommand{\positive}[1]{\textcolor{positivegreen}{\textbf{#1}}}
\newcommand{\negative}[1]{\textcolor{negativered}{\textbf{#1}}}

% =============================================================================
% CUSTOM BEAMER ENVIRONMENTS
% =============================================================================

% definitionbox[Título]: navy-bordered titled box for formal definitions
\newenvironment{definitionbox}[1]{%
  \begin{tcolorbox}[
    title={#1},
    fonttitle=\bfseries,
    colbacktitle=primarynavy,
    coltitle=white,
    colback=primarynavy!6,
    colframe=primarynavy,
    boxrule=1.5pt,
    arc=4pt,
    left=8pt, right=8pt, top=4pt, bottom=6pt
  ]
}{%
  \end{tcolorbox}
}

% keybox: teal background box for key insights & takeaways
\newenvironment{keybox}{%
  \begin{tcolorbox}[
    colback=warmteal!10,
    colframe=warmteal,
    boxrule=0pt,
    leftrule=4pt,
    arc=3pt,
    left=8pt, right=8pt, top=6pt, bottom=6pt
  ]
}{%
  \end{tcolorbox}
}

% highlightbox: amber accent box for alerts & emphasis
\newenvironment{highlightbox}{%
  \begin{tcolorbox}[
    colback=accentamber!8,
    colframe=accentamber,
    boxrule=0pt,
    leftrule=4pt,
    arc=3pt,
    left=8pt, right=8pt, top=6pt, bottom=6pt
  ]
}{%
  \end{tcolorbox}
}

% methodbox: navy left-accent box for methods & mechanisms
\newenvironment{methodbox}{%
  \begin{tcolorbox}[
    colback=primarynavy!6,
    colframe=primarynavy,
    boxrule=0pt,
    leftrule=4pt,
    arc=3pt,
    left=8pt, right=8pt, top=6pt, bottom=6pt
  ]
}{%
  \end{tcolorbox}
}

% quotebox: italic amber-accent box for economic quotes
\newenvironment{quotebox}{%
  \begin{tcolorbox}[
    colback=accentamber!6,
    colframe=accentamber!50,
    boxrule=0pt,
    leftrule=4pt,
    arc=0pt,
    left=14pt, right=8pt, top=6pt, bottom=6pt,
    fontupper=\itshape
  ]
}{%
  \end{tcolorbox}
}

% resultbox: amber-bordered box for empirical results
\newenvironment{resultbox}{%
  \begin{tcolorbox}[
    colback=accentamber!12,
    colframe=accentamber,
    boxrule=2pt,
    arc=4pt,
    left=8pt, right=8pt, top=6pt, bottom=6pt
  ]
}{%
  \end{tcolorbox}
}

% =============================================================================
% BEAMER THEME CONFIGURATION
% =============================================================================

\usetheme{default}
\usecolortheme{default}
\setbeamertemplate{navigation symbols}{}
\setbeamertemplate{footline}[frame number]

% Core colors
\setbeamercolor{frametitle}{fg=primarynavy, bg=white}
\setbeamercolor{title}{fg=primarynavy}
\setbeamercolor{subtitle}{fg=warmteal}
\setbeamercolor{author}{fg=primarynavy}
\setbeamercolor{date}{fg=accentamber}
\setbeamercolor{structure}{fg=primarynavy}
\setbeamercolor{alerted text}{fg=accentamber}

% Bullet point colors
\setbeamercolor{itemize item}{fg=primarynavy}
\setbeamercolor{itemize subitem}{fg=warmteal}
\setbeamertemplate{itemize item}{$\bullet$}
\setbeamertemplate{itemize subitem}{$\rightarrow$}

% Frame title with amber underline
\setbeamertemplate{frametitle}{%
  \vspace{0.4em}%
  {\color{primarynavy}\textbf{\insertframetitle}}%
  \vspace{0.15em}%
  \newline{\color{accentamber}\rule{\textwidth}{0.5pt}}%
}

% Block environments
\setbeamercolor{block title}{bg=primarynavy!90, fg=white}
\setbeamercolor{block body}{bg=primarynavy!8, fg=black}
\setbeamercolor{block title alerted}{bg=accentamber!80, fg=white}
\setbeamercolor{block body alerted}{bg=accentamber!10, fg=black}


% -----------------------------------------------------------------------------
\title{Introducción al Desarrollo Económico}
\subtitle{Economía del Desarrollo | Clase 1}
\author{[Nombre del Profesor]}
\institute{Departamento de Economía\\Universidad de Santiago de Chile}
\date{Semestre 1, 2026}
% -----------------------------------------------------------------------------

\begin{document}

% ============================================================
% SLIDE 1: Portada
% ============================================================
\begin{frame}[plain]
  \titlepage
\end{frame}

% ============================================================
% BLOQUE 0: APERTURA — LAS TRES PREGUNTAS
% ============================================================

% ============================================================
% SLIDE 2: Las tres preguntas
% ============================================================
\begin{frame}{Las tres preguntas del desarrollo}
  \begin{keybox}
    \begin{enumerate}
      \item \textbf{¿Por qué algunos países son ricos y otros son pobres?}\\[0.2em]
            \muted{El origen del crecimiento económico sostenido.}
      \vspace{0.5em}
      \item \textbf{¿Por qué hay tanta desigualdad entre países?}\\[0.2em]
            \muted{El papel de la geografía, las instituciones y la historia.}
      \vspace{0.5em}
      \item \textbf{¿Por qué algunos países no logran despegar?}\\[0.2em]
            \muted{Las trampas de pobreza y las fallas de mercado.}
    \end{enumerate}
  \end{keybox}
  \vspace{0.5em}
  \centering
  \muted{Este curso es un intento sistemático de responder estas tres preguntas.}
\end{frame}

% ============================================================
% SLIDE 3: Hoja de ruta (TikZ)
% ============================================================
\begin{frame}{Hoja de ruta: 12 clases, 3 misterios}
  \begin{center}
  \begin{tikzpicture}[
    box/.style={draw=primarynavy, fill=lightgray, rounded corners=4pt,
                minimum width=3.5cm, minimum height=2.6cm, align=center,
                font=\small},
    arrow/.style={-{Stealth[length=8pt]}, line width=2pt, color=accentamber},
    node distance=1.0cm
  ]
    \node[box] (part1) {
      \textbf{\color{primarynavy}Parte I}\\[4pt]
      \color{warmteal}\textit{El misterio del}\\
      \color{warmteal}\textit{crecimiento}\\[6pt]
      \scriptsize L2 Malthus\\
      \scriptsize L3 Solow\\
      \scriptsize L4 Ideas
    };
    \node[box, right=1.4cm of part1] (part2) {
      \textbf{\color{primarynavy}Parte II}\\[4pt]
      \color{warmteal}\textit{El misterio de la}\\
      \color{warmteal}\textit{desigualdad}\\[6pt]
      \scriptsize L5 Geografía\\
      \scriptsize L6 Instituciones\\
      \scriptsize L7 Cultura · L8 Comercio
    };
    \node[box, right=1.4cm of part2] (part3) {
      \textbf{\color{primarynavy}Parte III}\\[4pt]
      \color{warmteal}\textit{El misterio del}\\
      \color{warmteal}\textit{estancamiento}\\[6pt]
      \scriptsize L9 Trampas\\
      \scriptsize L10 Salud/Edu\\
      \scriptsize L11 Crédito · L12 Gob.
    };
    \draw[arrow] (part1.east) -- (part2.west);
    \draw[arrow] (part2.east) -- (part3.west);
  \end{tikzpicture}
  \end{center}
  \vspace{0.2em}
  \centering
  \muted{\small Textbooks: \citet{AshrafWeil2025_growth} (Partes I--II) · \citet{Ray1998_development} (Parte III)}
\end{frame}

% ============================================================
% SLIDE 4: Lo que la economía puede (y no puede) explicar
% ============================================================
\begin{frame}{Lo que la economía puede (y no puede) explicar}
  \begin{columns}[T]
    \column{0.50\textwidth}
      \textbf{\color{positivegreen}Lo que podemos hacer:}
      \begin{itemize}
        \item Identificar \key{mecanismos} que generan crecimiento
        \item Cuantificar el peso relativo de cada factor
        \item Diseñar y evaluar \key{políticas} con rigor causal
        \item Entender por qué algunos países logran escapar
      \end{itemize}
    \column{0.46\textwidth}
      \begin{highlightbox}
        \textbf{Lo que \emph{no} podemos hacer:}
        \begin{itemize}
          \item Dar una receta única de desarrollo
          \item Ignorar la historia, la política y la cultura
          \item Predecir el futuro de ningún país con certeza
        \end{itemize}
      \end{highlightbox}
  \end{columns}
  \vspace{0.6em}
  \begin{center}
    \muted{La economía del desarrollo es una ciencia empírica: los datos disciplinan las teorías.}
  \end{center}
\end{frame}

% ============================================================
% SLIDE 5: Por qué importa
% ============================================================
\begin{frame}{¿Por qué importa esto?}
  \begin{quotebox}
    ``For the first time in history, the escape from poverty is not limited
    to a lucky few. The question is whether the escape will continue,
    and how fast.''
    \begin{flushright}--- \citet{Deaton2013_escape}\end{flushright}
  \end{quotebox}
  \vspace{0.3em}
  \begin{itemize}
    \item Mil millones de personas salieron de la pobreza extrema entre 1990 y 2020
    \item \positive{Chile} pasó de ingreso medio-bajo en 1980 a ingreso alto en 2022
    \item Pero \negative{más de 80 millones} siguen en pobreza extrema solo en América Latina \muted{\scriptsize [CEPAL, 2023]}
    \item La brecha entre el país más rico y el más pobre: \key{80:1} en ingreso per cápita
  \end{itemize}
  \vspace{0.3em}
  \centering
  \muted{Entender el desarrollo no es solo académico: es urgente.}
\end{frame}

% ============================================================
% BLOQUE 1: HECHOS ESTILIZADOS
% ============================================================
\begin{frame}[plain]
  \begin{center}
    \vspace{1.5em}
    {\Large \color{primarynavy}\textbf{Bloque A: Los hechos estilizados}}\\[0.6em]
    {\large \color{warmteal}\textit{El misterio en datos}}
  \end{center}
\end{frame}

% ============================================================
% SLIDE 6: La brecha de ingresos hoy
% ============================================================
\begin{frame}{La brecha de ingresos hoy (2022)}
  \begin{columns}[T]
    \column{0.52\textwidth}
      \IfFileExists{../Figures/fig_income_bar.pdf}{%
        \includegraphics[width=\linewidth]{../Figures/fig_income_bar}%
      }{%
        \begin{center}\vspace{0.8em}
          \color{slategray}\small
          [Figura: Top/Bottom 10 países por PIB pc PPA (2022)]\\[0.4em]
          \texttt{Ejecutar scripts/R/lecture01\_figures.R}
          \vspace{0.8em}
        \end{center}%
      }
    \column{0.44\textwidth}
      \begin{itemize}
        \item Luxemburgo: \textasciitilde\$130{,}000 PPP per cápita
        \item Estados Unidos: \textasciitilde\$80{,}000 PPP per cápita
        \item \positive{Chile}: \textasciitilde\$28{,}000 PPP per cápita
        \item Nigeria: \textasciitilde\$6{,}000 PPP per cápita
        \item \negative{Malaui}: \textasciitilde\$1{,}600 PPP per cápita
      \end{itemize}
      \vspace{0.4em}
      \begin{resultbox}
        Razón rico/pobre: \textbf{\textasciitilde80:1}\\[0.2em]
        \muted{\small La misma persona produce 80 veces más en Luxemburgo que en Malaui.}
      \end{resultbox}
      \vspace{0.3em}
      \muted{\small Fuente: World Development Indicators, Banco Mundial (2022).}
  \end{columns}
\end{frame}

% ============================================================
% SLIDE 7: Divergencia en el tiempo
% ============================================================
\begin{frame}{Divergencia en el tiempo: 1820--2022}
  \begin{columns}[T]
    \column{0.56\textwidth}
      \IfFileExists{../Figures/fig_maddison_lines.pdf}{%
        \includegraphics[width=\linewidth]{../Figures/fig_maddison_lines}%
      }{%
        \begin{center}\vspace{0.8em}
          \color{slategray}\small
          [Figura: series Maddison 1820--2022, países seleccionados]\\[0.4em]
          \texttt{Ejecutar scripts/R/lecture01\_figures.R}
          \vspace{0.8em}
        \end{center}%
      }
    \column{0.40\textwidth}
      \begin{highlightbox}
        \textbf{``Divergence, Big Time''}\\
        \muted{\small \citet{Pritchett1997_divergence}}
      \end{highlightbox}
      \vspace{0.5em}
      \begin{itemize}
        \item En 1820, la brecha rica/pobre era \textasciitilde\textbf{3:1}
        \item En 2022, la brecha es \textbf{80:1}
        \item La divergencia se aceleró desde la Revolución Industrial
        \item Algunos países despegaron; la mayoría, no
      \end{itemize}
      \vspace{0.3em}
      \muted{\small Fuente: Maddison Project Database 2023.}
  \end{columns}
\end{frame}

% ============================================================
% SLIDE 8: El mundo en 1820
% ============================================================
\begin{frame}{El mundo en 1820: pobreza casi universal}
  \begin{columns}[T]
    \column{0.54\textwidth}
      \IfFileExists{../Figures/fig_maddison_dist_1820.pdf}{%
        \includegraphics[width=\linewidth]{../Figures/fig_maddison_dist_1820}%
      }{%
        \begin{center}\vspace{0.8em}
          \color{slategray}\small
          [Figura: distribución global de ingresos, Maddison 1820]\\[0.4em]
          \texttt{Ejecutar scripts/R/lecture01\_figures.R}
          \vspace{0.8em}
        \end{center}%
      }
    \column{0.42\textwidth}
      \begin{itemize}
        \item Ingreso promedio global: \textasciitilde\$1{,}100 (PPP 2011)
        \item La distribución era \key{casi unimodal}:\\
              todos eran pobres
        \item Incluso Europa Occidental apenas superaba \$2{,}000
        \item La pobreza extrema era la condición \emph{normal} de la humanidad
      \end{itemize}
      \vspace{0.5em}
      \muted{\small Malthus describía este mundo: el ingreso per cápita tendía siempre al nivel de subsistencia. Próxima clase: ¿por qué?}
  \end{columns}
\end{frame}

% ============================================================
% SLIDE 9: El mundo en 2022
% ============================================================
\begin{frame}{El mundo en 2022: divergencia bimodal}
  \begin{columns}[T]
    \column{0.54\textwidth}
      \IfFileExists{../Figures/fig_maddison_dist_2022.pdf}{%
        \includegraphics[width=\linewidth]{../Figures/fig_maddison_dist_2022}%
      }{%
        \begin{center}\vspace{0.8em}
          \color{slategray}\small
          [Figura: distribución global de ingresos, Maddison 2022]\\[0.4em]
          \texttt{Ejecutar scripts/R/lecture01\_figures.R}
          \vspace{0.8em}
        \end{center}%
      }
    \column{0.42\textwidth}
      \begin{itemize}
        \item La distribución es ahora \key{bimodal}:
          \begin{itemize}
            \item Un grupo escapó hacia altos ingresos
            \item Otro grupo sigue rezagado
          \end{itemize}
        \item Buenas noticias: Asia está llenando el centro
        \item \positive{Chile} está cerca del umbral de ingreso alto
      \end{itemize}
      \vspace{0.4em}
      \muted{\small \citet{SalaiMartin2006_income}: la clase media global existe, pero es frágil.}
  \end{columns}
\end{frame}

% ============================================================
% SLIDE 10: Salud y riqueza
% ============================================================
\begin{frame}{El Gran Escape: riqueza y salud van juntos}
  \begin{columns}[T]
    \column{0.54\textwidth}
      \IfFileExists{../Figures/fig_income_life_expectancy.pdf}{%
        \includegraphics[width=\linewidth]{../Figures/fig_income_life_expectancy}%
      }{%
        \begin{center}\vspace{0.8em}
          \color{slategray}\small
          [Figura: scatter log PIB pc vs. esperanza de vida, WDI 2022]\\[0.4em]
          \texttt{Ejecutar scripts/R/lecture01\_figures.R}
          \vspace{0.8em}
        \end{center}%
      }
    \column{0.42\textwidth}
      \begin{itemize}
        \item Correlación fuerte: más ingreso $\rightarrow$ más vida
        \item Singapur: \$100k PPP, 84 años
        \item Sierra Leona: \$2k PPP, 55 años
        \item \positive{Chile}: \$28k PPP, 80 años
      \end{itemize}
      \vspace{0.4em}
      \begin{quotebox}
        \small El Gran Escape es simultáneamente de la pobreza y de la muerte prematura.
        \begin{flushright}---\citet{Deaton2013_escape}\end{flushright}
      \end{quotebox}
  \end{columns}
\end{frame}

% ============================================================
% SLIDE 11: Educación y riqueza
% ============================================================
\begin{frame}{Educación y riqueza: otro correlato poderoso}
  \begin{columns}[T]
    \column{0.50\textwidth}
      \begin{center}
      \begin{tikzpicture}[scale=0.70]
        \draw[->] (0,0) -- (6.2,0) node[right]{\scriptsize PIB pc (log)};
        \draw[->] (0,0) -- (0,4.2) node[above]{\scriptsize Escolaridad};
        \foreach \x/\y/\c in {
          0.4/0.8/negativered,
          0.7/1.1/negativered,
          1.0/1.4/negativered,
          1.4/1.8/slategray,
          2.0/2.3/slategray,
          2.5/2.7/slategray,
          3.0/3.0/warmteal,
          3.6/3.2/warmteal,
          4.2/3.5/warmteal,
          4.8/3.7/positivegreen,
          5.4/3.9/positivegreen%
        }{
          \fill[\c, opacity=0.65] (\x,\y) circle (3pt);
        }
        \fill[accentamber] (3.4,3.1) circle (4.5pt)
          node[above right, font=\scriptsize]{\textbf{Chile}};
        \draw[primarynavy, thick, dashed] (0.2,0.5) -- (5.8,4.0);
      \end{tikzpicture}
      \end{center}
    \column{0.46\textwidth}
      \begin{itemize}
        \item Países ricos invierten más en educación
        \item ¿La educación \emph{causa} el crecimiento o al revés?
        \item Esta pregunta estructurará la Parte III
      \end{itemize}
      \vspace{0.5em}
      \begin{highlightbox}
        \small \textbf{Para pensar:}\\
        \small ¿Invierte Chile suficiente en edu\-ca\-ción dado su nivel de ingreso?
      \end{highlightbox}
  \end{columns}
\end{frame}

% ============================================================
% SLIDE 12: La distribución global de ingresos
% ============================================================
\begin{frame}{La distribución global de ingresos: de unimodal a bimodal}
  \begin{columns}[T]
    \column{0.52\textwidth}
      \begin{center}
      \begin{tikzpicture}[scale=0.78]
        \draw[->] (0,0) -- (7.0,0) node[right]{\scriptsize Ingreso};
        \draw[->] (0,0) -- (0,3.8) node[above]{\scriptsize Densidad};
        % 1970: unimodal, low income
        \draw[warmteal!70, thick, dashed]
          plot[smooth, tension=0.8] coordinates {
            (0.2,0.05)(0.6,0.4)(1.2,1.8)(2.0,0.8)(2.8,0.2)(3.8,0.03)
          };
        \node[warmteal!90, font=\footnotesize\itshape] at (1.4,2.0) {1970};
        % 2020: bimodal
        \draw[primarynavy, thick]
          plot[smooth, tension=0.8] coordinates {
            (0.2,0.05)(0.7,0.35)(1.3,0.9)(1.8,0.5)(2.4,0.35)
            (3.5,1.1)(4.4,2.6)(5.1,1.0)(6.0,0.1)
          };
        \node[primarynavy, font=\footnotesize\bfseries] at (4.6,2.8) {2020};
        % Chile marker
        \draw[accentamber, thick, ->] (3.7,-0.35)
          node[below, font=\scriptsize]{\textbf{Chile}} -- (3.9,0.2);
      \end{tikzpicture}
      \end{center}
    \column{0.44\textwidth}
      \begin{itemize}
        \item En 1970: distribución casi unimodal de bajos ingresos
        \item En 2020: \key{bimodal} --- dos ``montañas''
        \item La primera montaña (pobres) se vacía lentamente
        \item La segunda (clase media-alta) crece, especialmente Asia
      \end{itemize}
      \vspace{0.3em}
      \muted{\small \citet{SalaiMartin2006_income}: la pobreza cae, pero la polarización persiste.}
  \end{columns}
\end{frame}

% ============================================================
% SLIDE 13: Chile en perspectiva
% ============================================================
\begin{frame}{Chile en perspectiva global}
  \begin{columns}[T]
    \column{0.54\textwidth}
      \IfFileExists{../Figures/fig_chile_maddison.pdf}{%
        \includegraphics[width=\linewidth]{../Figures/fig_chile_maddison}%
      }{%
        \begin{center}\vspace{0.8em}
          \color{slategray}\small
          [Figura: Chile vs. pares LA + promedio mundial, Maddison 1950--2022]\\[0.4em]
          \texttt{Ejecutar scripts/R/lecture01\_figures.R}
          \vspace{0.8em}
        \end{center}%
      }
    \column{0.42\textwidth}
      \begin{itemize}
        \item Chile \positive{cruzó} el umbral de ingreso alto (BM) en 2022
        \item Crecimiento promedio 1990--2012: \textbf{\textasciitilde5\%} anual
        \item Desde 2012: \negative{desaceleración} a \textasciitilde1.8\%
        \item ¿Riesgo de \key{trampa de ingreso medio}?
      \end{itemize}
      \vspace{0.4em}
      \begin{keybox}
        \small Chile es nuestro \textbf{caso base} durante todo el semestre. Cada hecho global se ancla en la experiencia chilena.
      \end{keybox}
  \end{columns}
\end{frame}

% ============================================================
% SLIDE 14: América Latina
% ============================================================
\begin{frame}{América Latina: ¿maldición o elección?}
  \begin{columns}[T]
    \column{0.52\textwidth}
      \begin{itemize}
        \item AL creció más lento que Asia del Este en los últimos 40 años
        \item \textbf{1960--1980}: crecimiento al 5\%+ (industrialización)
        \item \textbf{``Década perdida'' 1980}: crisis de deuda, ajuste
        \item \textbf{1990--2003}: reformas + crecimiento moderado
        \item \textbf{2003--2013}: boom de commodities
        \item \textbf{2014--hoy}: \negative{estancamiento relativo}
      \end{itemize}
    \column{0.44\textwidth}
      \begin{highlightbox}
        \textbf{La paradoja latinoamericana:}\\[0.3em]
        \begin{itemize}
          \item Rica en recursos naturales
          \item Con instituciones democráticas
          \item Pero con baja productividad y alta desigualdad
        \end{itemize}
        \vspace{0.3em}
        \small Parte II explorará qué hay detrás.
      \end{highlightbox}
  \end{columns}
\end{frame}

% ============================================================
% SLIDE 15: Cinco hechos estilizados
% ============================================================
\begin{frame}{Cinco hechos estilizados sobre el desarrollo}
  \begin{enumerate}
    \setlength{\itemsep}{0.15em}
    \item \textbf{La brecha de ingresos es enorme:} razón \key{80:1} entre el país más rico y el más pobre.
    \item \textbf{La brecha ha crecido históricamente:} en 1820 era 3:1; hoy es 80:1 (\textit{``Divergence, Big Time''}).
    \item \textbf{El ingreso se correlaciona con todo lo bueno:} salud, educación, seguridad, democracia.
    \item \textbf{El escape es posible:} Asia del Este y \positive{Chile} lo demostraron.
    \item \textbf{El estancamiento también es posible:} muchos países llevan décadas sin converger.
  \end{enumerate}
  \vspace{0.3em}
  \centering
  \muted{Estos hechos exigen explicación. El resto del curso es ese intento.}
\end{frame}

% ============================================================
% SLIDE 16: El gran rompecabezas
% ============================================================
\begin{frame}{El gran rompecabezas}
  \begin{center}
  \begin{tikzpicture}[
    q/.style={draw=accentamber, fill=accentamber!10, rounded corners=3pt,
              minimum width=4.2cm, minimum height=0.9cm, align=center,
              font=\small\itshape},
    a/.style={draw=primarynavy, fill=lightgray, rounded corners=3pt,
              minimum width=4.2cm, minimum height=0.9cm, align=center,
              font=\small\bfseries, text=primarynavy},
    node distance=0.55cm
  ]
    \node[q] (q1) {¿Por qué crece la economía?};
    \node[q, below=of q1] (q2) {¿Por qué hay tanta desigualdad?};
    \node[q, below=of q2] (q3) {¿Por qué no todos despegan?};
    \node[a, right=2.8cm of q1] (a1) {L2--L4: Malthus, Solow, Romer};
    \node[a, right=2.8cm of q2] (a2) {L5--L8: Geografía, Inst., Cultura};
    \node[a, right=2.8cm of q3] (a3) {L9--L12: Trampas, Salud, Crédito};
    \draw[->, warmteal, thick] (q1.east) -- (a1.west);
    \draw[->, warmteal, thick] (q2.east) -- (a2.west);
    \draw[->, warmteal, thick] (q3.east) -- (a3.west);
  \end{tikzpicture}
  \end{center}
  \vspace{0.4em}
  \centering
  \muted{Ninguna teoría única responde las tres preguntas. Necesitamos múltiples lentes.}
\end{frame}

% ============================================================
% SLIDE 17: Vista previa Parte I
% ============================================================
\begin{frame}{Parte I: El misterio del crecimiento (L2--L4)}
  \begin{columns}[T]
    \column{0.31\textwidth}
      \begin{methodbox}
        \textbf{L2: Malthus}\\[0.3em]
        \small La trampa maltusiana. ¿Por qué el crecimiento per cápita era imposible durante milenios? La transición demográfica.
      \end{methodbox}
    \column{0.31\textwidth}
      \begin{methodbox}
        \textbf{L3: Solow}\\[0.3em]
        \small Capital, trabajo y TFP. Convergencia condicional. El modelo MRW. ¿Por qué los países convergen (o no)?
      \end{methodbox}
    \column{0.31\textwidth}
      \begin{methodbox}
        \textbf{L4: Ideas}\\[0.3em]
        \small Crecimiento endógeno (Romer). Ideas como bien no rival. Destrucción creativa (Schumpeter).
      \end{methodbox}
  \end{columns}
  \vspace{0.5em}
  \centering
  \muted{\small Textbook: \citet{AshrafWeil2025_growth}, capítulos 1--6}
\end{frame}

% ============================================================
% SLIDE 18: Vista previa Parte II
% ============================================================
\begin{frame}{Parte II: El misterio de la desigualdad (L5--L8)}
  \begin{columns}[T]
    \column{0.23\textwidth}
      \begin{methodbox}
        \textbf{L5}\\Geografía\\[0.3em]
        \small Sachs, Gallup. Trópicos, costas, malaria y el destino de las naciones.
      \end{methodbox}
    \column{0.23\textwidth}
      \begin{methodbox}
        \textbf{L6}\\Instituciones\\[0.3em]
        \small Acemoglu et al. Colonialismo, mortalidad de colonizadores y IV.
      \end{methodbox}
    \column{0.23\textwidth}
      \begin{methodbox}
        \textbf{L7}\\Cultura\\[0.3em]
        \small Weber, confianza social, Tabellini. ¿Importa la cultura?
      \end{methodbox}
    \column{0.23\textwidth}
      \begin{methodbox}
        \textbf{L8}\\Comercio\\[0.3em]
        \small Apertura, ISI, los tigres asiáticos. Comercio y desarrollo.
      \end{methodbox}
  \end{columns}
  \vspace{0.4em}
  \centering
  \muted{\small \textbf{Pregunta central:} ¿Por qué países con recursos similares tienen trayectorias de desarrollo tan distintas?}
\end{frame}

% ============================================================
% SLIDE 19: Vista previa Parte III
% ============================================================
\begin{frame}{Parte III: El misterio del estancamiento (L9--L12)}
  \begin{columns}[T]
    \column{0.23\textwidth}
      \begin{methodbox}
        \textbf{L9}\\Trampas\\[0.3em]
        \small Big push. Equilibrios múltiples. Curvas en S.
      \end{methodbox}
    \column{0.23\textwidth}
      \begin{methodbox}
        \textbf{L10}\\Salud/Edu\\[0.3em]
        \small RCTs. Kremer, Duflo. Desparasitación. Capital humano.
      \end{methodbox}
    \column{0.23\textwidth}
      \begin{methodbox}
        \textbf{L11}\\Crédito\\[0.3em]
        \small Fallas de mercado. Microfinanzas. Evaluaciones de impacto.
      \end{methodbox}
    \column{0.23\textwidth}
      \begin{methodbox}
        \textbf{L12}\\Gobierno\\[0.3em]
        \small Corrupción, ayuda externa, el debate Sachs vs.\ Easterly.
      \end{methodbox}
  \end{columns}
  \vspace{0.5em}
  \muted{\small Textbook: \citet{Ray1998_development} + artículos de Banerjee \& Duflo}
\end{frame}

% ============================================================
% SLIDE 20: Advertencia metodológica
% ============================================================
\begin{frame}{Advertencia metodológica fundamental}
  \begin{highlightbox}
    \textbf{El problema de causalidad en economía del desarrollo:}\\[0.4em]
    Los países ricos tienen mejores instituciones, más educación, mejor salud, mayor apertura comercial\ldots\\[0.3em]
    Pero \textbf{¿qué causa qué?}
  \end{highlightbox}
  \vspace{0.5em}
  \begin{columns}[T]
    \column{0.47\textwidth}
      \textbf{Los problemas:}
      \begin{itemize}
        \item Causalidad inversa ($Y \rightarrow X$ además de $X \rightarrow Y$)
        \item Variable omitida (un tercer factor causa ambos)
        \item Sesgo de selección
      \end{itemize}
    \column{0.49\textwidth}
      \textbf{Las herramientas que usaremos:}
      \begin{itemize}
        \item Variables instrumentales (IV)
        \item Experimentos aleatorios (RCT)
        \item Diferencias en diferencias (DiD)
      \end{itemize}
  \end{columns}
  \vspace{0.4em}
  \centering
  \muted{La Parte III muestra la frontera metodológica del campo.}
\end{frame}

% ============================================================
% BLOQUE 2: FILOSOFÍA DE EVALUACIÓN
% ============================================================
\begin{frame}[plain]
  \begin{center}
    \vspace{1.5em}
    {\Large \color{primarynavy}\textbf{Bloque B: Evaluación y organización}}\\[0.6em]
    {\large \color{warmteal}\textit{Filosofía del curso y Country Portfolio}}
  \end{center}
\end{frame}

% ============================================================
% SLIDE 21: Estructura del curso
% ============================================================
\begin{frame}{Estructura del curso}
  \begin{center}
  \small
  \begin{tabular}{@{}lllp{5.8cm}@{}}
    \toprule
    \textbf{Parte} & \textbf{Clases} & \textbf{Textbook} & \textbf{Tema} \\
    \midrule
    I & L1--L4 & Ashraf \& Weil, Caps.\ 1--6 & El misterio del crecimiento \\[0.2em]
    II & L5--L8 & Ashraf \& Weil, Caps.\ 10--12 & El misterio de la desigualdad \\[0.2em]
    III & L9--L12 & Ray + artículos & El misterio del estancamiento \\
    \bottomrule
  \end{tabular}
  \end{center}
  \vspace{0.5em}
  \begin{columns}[T]
    \column{0.50\textwidth}
      \textbf{Lecturas requeridas por clase:}
      \begin{itemize}
        \item \citet{AshrafWeil2025_growth} (textbook principal)
        \item \citet{Ray1998_development} (Parte III)
        \item Artículos en SIDING (acceso abierto)
      \end{itemize}
    \column{0.46\textwidth}
      \begin{keybox}
        \small Se espera que hayan leído \textbf{antes de clase}. Las clases profundizan, no solo repiten el texto.
      \end{keybox}
  \end{columns}
\end{frame}

% ============================================================
% SLIDE 22: Evaluación resumen
% ============================================================
\begin{frame}{Sistema de evaluación}
  \begin{center}
  \small
  \begin{tabular}{@{}p{5.0cm}rp{4.5cm}@{}}
    \toprule
    \textbf{Componente} & \textbf{Peso} & \textbf{Descripción} \\
    \midrule
    Tareas (T1--T8, descarta la peor) & 35\% & Take-home, 7 días, IA permitida \\[0.2em]
    Control 1 (fin Parte I)           &  8\% & En sala, sin IA, 60 min \\[0.2em]
    Control 2 (fin Parte II)          &  8\% & En sala, sin IA, 60 min \\[0.2em]
    Control 3 (fin Parte III)         &  9\% & En sala, sin IA, 60 min \\[0.2em]
    Examen final                      & 30\% & Acumulativo, sin IA \\[0.2em]
    Participación                     & 10\% & Discusión activa en clases \\
    \midrule
    \textbf{Total}                              & \textbf{100\%} & \\
    \bottomrule
  \end{tabular}
  \end{center}
\end{frame}

% ============================================================
% SLIDE 23: Tareas semanales
% ============================================================
\begin{frame}{Tareas semanales: filosofía y reglas}
  \begin{columns}[T]
    \column{0.54\textwidth}
      \begin{keybox}
        \textbf{Diseñadas para el aprendizaje activo:}
        \begin{itemize}
          \item 8 tareas en total; se descarta la peor
          \item Plazo: 7 días desde la clase correspondiente
          \item Entrega: PDF via SIDING
        \end{itemize}
      \end{keybox}
      \vspace{0.5em}
      \textbf{Sobre el uso de inteligencia artificial:}
      \begin{itemize}
        \item \positive{Bienvenida} como herramienta de apoyo
        \item \textbf{Obligatorio:} mostrar razonamiento propio
        \item Si usas IA: declarar en la tarea cómo la usaste
      \end{itemize}
    \column{0.42\textwidth}
      \begin{highlightbox}
        \textbf{¿Por qué permitimos IA?}\\[0.3em]
        \small Porque en el mundo real los economistas usan todas las herramientas disponibles. Lo que evaluamos es el \emph{razonamiento económico}, no la memorización.
      \end{highlightbox}
  \end{columns}
\end{frame}

% ============================================================
% SLIDE 24: Controles y examen
% ============================================================
\begin{frame}{Controles y examen: evaluación sin red}
  \begin{columns}[T]
    \column{0.48\textwidth}
      \textbf{Controles (uno por Parte):}
      \begin{itemize}
        \item 60 minutos, en sala de clases
        \item Sin apuntes, sin IA, sin internet
        \item Preguntas cortas + análisis de datos
        \item Fechas: al cierre de cada Parte
      \end{itemize}
      \vspace{0.4em}
      \textbf{Examen final:}
      \begin{itemize}
        \item 120 minutos, fecha oficial USACH
        \item Cubre todo el semestre
        \item Énfasis en síntesis y argumentación
      \end{itemize}
    \column{0.48\textwidth}
      \begin{methodbox}
        \textbf{¿Por qué evaluamos sin IA?}\\[0.3em]
        \small Queremos que sean capaces de \emph{razonar en tiempo real} sin apoyo externo. Los controles y el examen evalúan comprensión genuina del material.
      \end{methodbox}
  \end{columns}
\end{frame}

% ============================================================
% SLIDE 25: Country Portfolio
% ============================================================
\begin{frame}{Country Portfolio: el hilo conductor del semestre}
  \begin{center}
  \begin{tikzpicture}[
    task/.style={draw=primarynavy, fill=lightgray, rounded corners=3pt,
                 minimum width=2.7cm, minimum height=1.5cm, align=center,
                 font=\scriptsize},
    arrow/.style={-{Stealth[length=6pt]}, line width=1.5pt, color=accentamber},
    node distance=0.6cm
  ]
    \node[task] (t1) {\textbf{T1}\\\small Perfil\\de país\\(tras L1)};
    \node[task, right=0.9cm of t1] (t3) {\textbf{T3}\\\small Diagnóstico\\Solow\\(tras L3--L4)};
    \node[task, right=0.9cm of t3] (t6) {\textbf{T6}\\\small Auditoría\\institucional\\(tras L6)};
    \node[task, right=0.9cm of t6] (t8) {\textbf{T8}\\\small Pobreza\\y política\\(tras L9--L10)};
    \draw[arrow] (t1.east) -- (t3.west);
    \draw[arrow] (t3.east) -- (t6.west);
    \draw[arrow] (t6.east) -- (t8.west);
  \end{tikzpicture}
  \end{center}
  \vspace{0.4em}
  \begin{itemize}
    \item Eligen \textbf{un país en desarrollo} hoy y lo siguen todo el semestre
    \item Chile es el \key{caso base} para comparación en cada tarea
    \item Las 4 tareas del portafolio construyen un análisis integral de desarrollo
  \end{itemize}
  \vspace{0.2em}
  \muted{\small Al final del semestre serán ``expertos'' del desarrollo de su país.}
\end{frame}

% ============================================================
% SLIDE 26: Tarea 1
% ============================================================
\begin{frame}{Tarea 1: Perfil de país}
  \begin{columns}[T]
    \column{0.54\textwidth}
      \textbf{Instrucciones:}
      \begin{enumerate}
        \item Elijan su país de desarrollo (hoy)
        \item Creen un perfil de 2 páginas con:
          \begin{itemize}
            \item PIB per cápita PPP (1990--hoy)
            \item Esperanza de vida al nacer
            \item Tasa de escolaridad secundaria
            \item Un indicador a su elección
          \end{itemize}
        \item Comparen con Chile y un país OCDE
        \item Mínimo 4 gráficos de OWID o WDI
      \end{enumerate}
    \column{0.42\textwidth}
      \begin{keybox}
        \textbf{Herramientas:}\\[0.3em]
        \begin{itemize}
          \item \href{https://ourworldindata.org}{ourworldindata.org}
          \item \href{https://databank.worldbank.org}{World Bank WDI}
          \item R + ggplot2 o Excel (lo que prefieran)
        \end{itemize}
        \vspace{0.3em}
        \small Pueden usar IA para el código. Deben declarar su uso.
      \end{keybox}
      \vspace{0.3em}
      \muted{\small \textbf{Plazo:} [fecha --- 7 días desde hoy]}
  \end{columns}
\end{frame}

% ============================================================
% SLIDE 27: Cómo elegir tu país
% ============================================================
\begin{frame}{¿Cómo elegir tu país?}
  \begin{columns}[T]
    \column{0.50\textwidth}
      \textbf{\positive{Buenas opciones:}}
      \begin{itemize}
        \item \textbf{América Latina:} Bolivia, Perú, Colombia, Ecuador, Honduras, Guatemala
        \item \textbf{África Subsahariana:} Ghana, Etiopía, Ruanda, Tanzania, Senegal
        \item \textbf{Asia del Sur:} Bangladesh, Pakistán, Nepal
        \item \textbf{Asia emergente:} Vietnam, Indonesia, Myanmar
      \end{itemize}
    \column{0.46\textwidth}
      \begin{highlightbox}
        \textbf{Eviten:}
        \begin{itemize}
          \item Países OCDE (ya son desarrollados)
          \item Micro-estados (Mónaco, San Marino)
          \item Países sin datos confiables (conflictos activos)
          \item El mismo país que un compañero
        \end{itemize}
      \end{highlightbox}
      \vspace{0.4em}
      \muted{\small Registren su elección en el formulario de SIDING \textbf{antes del viernes}.}
  \end{columns}
\end{frame}

% ============================================================
% SLIDE 28: Recursos y contacto
% ============================================================
\begin{frame}{Recursos y contacto}
  \begin{columns}[T]
    \column{0.48\textwidth}
      \textbf{Bibliografía:}
      \begin{itemize}
        \item \citet{AshrafWeil2025_growth}: textbook principal
        \item \citet{Ray1998_development}: Parte III
        \item Artículos en SIDING (acceso abierto)
      \end{itemize}
      \vspace{0.4em}
      \textbf{Datos:}
      \begin{itemize}
        \item \href{https://ourworldindata.org}{ourworldindata.org}
        \item \href{https://databank.worldbank.org/source/world-development-indicators}{World Bank WDI}
        \item \href{https://www.rug.nl/ggdc/historicaldevelopment/maddison/}{Maddison Project 2023}
      \end{itemize}
    \column{0.48\textwidth}
      \begin{keybox}
        \textbf{Contacto:}\\[0.3em]
        Correo: \texttt{[email@usach.cl]}\\[0.2em]
        Horario de consulta:\\
        \texttt{[día], [hora]}\\
        Oficina: \texttt{[número]}
      \end{keybox}
      \vspace{0.4em}
      \begin{methodbox}
        \small SIDING: programa, lecturas, tareas, plantillas de datos.
      \end{methodbox}
  \end{columns}
\end{frame}

% ============================================================
% BLOQUE 3: HERRAMIENTAS BÁSICAS
% ============================================================
\begin{frame}[plain]
  \begin{center}
    \vspace{1.5em}
    {\Large \color{primarynavy}\textbf{Bloque C: Herramientas básicas}}\\[0.6em]
    {\large \color{warmteal}\textit{Instrumentos cuantitativos para el análisis del desarrollo}}
  \end{center}
\end{frame}

% ============================================================
% HERRAMIENTAS A: LOGARITMOS Y CRECIMIENTO
% ============================================================

% ============================================================
% SLIDE 29: ¿Por qué logaritmos?
% ============================================================
\begin{frame}{¿Por qué usamos logaritmos en economía del desarrollo?}
  \begin{columns}[T]
    \column{0.48\textwidth}
      \textbf{Escala lineal:}
      \begin{center}
      \begin{tikzpicture}[scale=0.72]
        \draw[->] (0,0) -- (4.8,0) node[right]{\tiny Año};
        \draw[->] (0,0) -- (0,3.8) node[above]{\tiny PIB};
        \draw[primarynavy, thick]
          plot[smooth] coordinates {(0,0.2)(1,0.4)(2,0.8)(3,1.6)(4,3.2)};
        \node[font=\tiny, text=slategray] at (2,-0.4) {Curva exponencial};
      \end{tikzpicture}
      \end{center}
      \small El crecimiento exponencial parece siempre acelerarse: difícil comparar tasas.
    \column{0.48\textwidth}
      \textbf{Escala logarítmica:}
      \begin{center}
      \begin{tikzpicture}[scale=0.72]
        \draw[->] (0,0) -- (4.8,0) node[right]{\tiny Año};
        \draw[->] (0,0) -- (0,3.8) node[above]{\tiny $\ln$(PIB)};
        \draw[accentamber, thick]
          plot[smooth] coordinates {(0,0.5)(1,1.0)(2,1.5)(3,2.0)(4,2.5)};
        \node[font=\tiny, text=slategray] at (2,-0.4) {Línea recta = crecimiento constante};
      \end{tikzpicture}
      \end{center}
      \small Una \key{línea recta} = tasa de crecimiento constante. La pendiente \emph{es} $g$.
  \end{columns}
  \vspace{0.4em}
  \begin{keybox}
    \small \textbf{Regla:} Si graficamos $\ln Y$ vs.\ tiempo y obtenemos una línea recta, la pendiente de esa línea \emph{es} la tasa de crecimiento porcentual.
  \end{keybox}
\end{frame}

% ============================================================
% SLIDE 30: La regla del 70
% ============================================================
\begin{frame}{La regla del 70}
  \begin{columns}[T]
    \column{0.54\textwidth}
      \begin{definitionbox}{Regla del 70}
        Si una economía crece al $g\%$ anual, su ingreso se \textbf{duplica} en aproximadamente:
        \[
          t^* \approx \frac{70}{g}\ \text{años}
        \]
      \end{definitionbox}
      \vspace{0.5em}
      \textbf{Ejemplos:}
      \begin{itemize}
        \item China ($g \approx 6\%$): dobla en \positive{12 años}
        \item Chile ($g \approx 3.5\%$): dobla en \positive{20 años}
        \item América Latina ($g \approx 1.5\%$): dobla en \negative{47 años}
      \end{itemize}
    \column{0.42\textwidth}
      \textbf{Derivación rápida:}\\[0.5em]
      Si $Y_t = Y_0 \cdot e^{gt}$, al doblar:
      \begin{align*}
        2 &= e^{g t^*} \\
        t^* &= \frac{\ln 2}{g} \approx \frac{0.693}{g} \approx \frac{70}{g\%}
      \end{align*}
      \vspace{0.3em}
      \begin{highlightbox}
        \small Pequeñas diferencias en $g$ producen \emph{enormes} diferencias en el largo plazo.
      \end{highlightbox}
  \end{columns}
\end{frame}

% ============================================================
% SLIDE 31: Diferencias logarítmicas = tasas de crecimiento
% ============================================================
\begin{frame}{Diferencias logarítmicas son tasas de crecimiento}
  \begin{columns}[T]
    \column{0.52\textwidth}
      \textbf{Resultado fundamental:}\\[0.5em]
      Si $Y_t = Y_{t-1}(1+g)$, entonces:
      \[
        g \approx \ln Y_t - \ln Y_{t-1} = \Delta \ln Y_t
      \]
      \begin{keybox}
        \small La diferencia logarítmica $\approx$ tasa de crecimiento porcentual.\\
        Aproximación muy buena para $g < 10\%$.
      \end{keybox}
    \column{0.44\textwidth}
      \textbf{¿Por qué es útil?}
      \begin{itemize}
        \item Propiedades estadísticas convenientes
        \item Permite \key{descomponer} multiplicativamente:
          \[
            Y = A K^\alpha L^\beta
          \]
          \[
            \ln Y = \ln A + \alpha \ln K + \beta \ln L
          \]
          \small Esto es la base de la \key{contabilidad del crecimiento} (Clase 3)
      \end{itemize}
  \end{columns}
\end{frame}

% ============================================================
% SLIDE 32: Crecimiento compuesto
% ============================================================
\begin{frame}{El poder del crecimiento compuesto}
  \begin{columns}[T]
    \column{0.50\textwidth}
      \begin{definitionbox}{Fórmula fundamental}
        \[
          Y_t = Y_0 \cdot (1 + g)^t \approx Y_0 \cdot e^{g t}
        \]
        $g$ = tasa de crecimiento anual, $t$ = años
      \end{definitionbox}
      \vspace{0.5em}
      \small Con $Y_0 = 100$:
      \begin{center}
      \small
      \begin{tabular}{@{}rrrr@{}}
        \toprule
        Años & $g=1\%$ & $g=2\%$ & $g=4\%$ \\
        \midrule
        10  & 110 & 122  & 148  \\
        25  & 128 & 164  & 267  \\
        50  & 165 & 269  & 711  \\
        100 & 271 & 724  & 5{,}050 \\
        \bottomrule
      \end{tabular}
      \end{center}
    \column{0.46\textwidth}
      \begin{highlightbox}
        \textbf{La diferencia de 2 puntos porcentuales:}\\[0.3em]
        \small Un país que crece al 4\% vs.\ uno que crece al 2\% durante 100 años termina con un PIB \textbf{7 veces mayor}.\\[0.3em]
        Esta es la respuesta a ``¿por qué importa tanto la tasa de crecimiento?''
      \end{highlightbox}
      \vspace{0.4em}
      \muted{\small Este resultado motiva todo el esfuerzo de identificar qué causa el crecimiento.}
  \end{columns}
\end{frame}

% ============================================================
% SLIDE 33: Ejercicio rápido
% ============================================================
\begin{frame}{Ejercicio rápido: Chile en dos escenarios}
  \begin{columns}[T]
    \column{0.48\textwidth}
      \textbf{Punto de partida:} $Y_{\text{CL},2025} = \$28{,}000$ PPP\\[0.5em]
      \begin{center}
      \small
      \begin{tabular}{@{}lrr@{}}
        \toprule
        Año  & $g = 2\%$ & $g = 4\%$ \\
        \midrule
        2025 & \$28{,}000 & \$28{,}000 \\
        2045 & \$41{,}600 & \$61{,}400 \\
        2065 & \$61{,}900 & \$133{,}600 \\
        \bottomrule
      \end{tabular}
      \end{center}
      \vspace{0.4em}
      \small En 40 años, la diferencia es de \negative{\$72{,}000 per cápita}.
    \column{0.48\textwidth}
      \begin{keybox}
        \textbf{Pregunta:}\\[0.3em]
        Chile creció al \textasciitilde5\% entre 1990--2012, pero solo al \textasciitilde1.8\% entre 2012--2024.\\[0.3em]
        ¿Qué implica esta desaceleración para el Chile de 2065?
      \end{keybox}
      \vspace{0.4em}
      \muted{\small Esta pregunta conecta directamente con los modelos de crecimiento en Parte I.}
  \end{columns}
\end{frame}

% ============================================================
% HERRAMIENTAS B: DATOS ENTRE PAÍSES
% ============================================================

% ============================================================
% SLIDE 34: PIB per cápita
% ============================================================
\begin{frame}{PIB per cápita: variantes y comparabilidad}
  \begin{columns}[T]
    \column{0.50\textwidth}
      \begin{definitionbox}{Nominales vs.\ PPP}
        \small
        \textbf{Nominal:} al tipo de cambio de mercado.\\[0.3em]
        \textbf{PPP} (Paridad de Poder de Compra): ajustado para que \$1 compre la misma canasta de bienes en todos los países.
      \end{definitionbox}
      \vspace{0.5em}
      \textbf{¿Cuándo usar cada uno?}
      \begin{itemize}
        \item \key{PPP}: comparar niveles de vida y bienestar
        \item Nominal: transacciones internacionales, deuda
        \item Usaremos \key{PPP} para todo lo que sea bienestar
      \end{itemize}
    \column{0.46\textwidth}
      \begin{highlightbox}
        \textbf{Ejemplo intuitivo:}\\[0.3em]
        \small Un corte de pelo en Bangladesh cuesta \$0.5 nominal. En EEUU: \$25 nominal.\\[0.2em]
        En PPP, el corte de pelo ``vale'' lo mismo en ambos países: el servicio es idéntico.\\[0.2em]
        $\Rightarrow$ El PIB PPP de Bangladesh \emph{sube} respecto al nominal.
      \end{highlightbox}
  \end{columns}
\end{frame}

% ============================================================
% SLIDE 35: Fuentes clave
% ============================================================
\begin{frame}{Fuentes clave de datos de desarrollo}
  \small
  \begin{tabular}{@{}p{2.6cm}p{4.0cm}p{4.8cm}@{}}
    \toprule
    \textbf{Fuente} & \textbf{Cobertura} & \textbf{Uso principal} \\
    \midrule
    \key{OWID} & 1800--hoy, 200+ países & Visualización rápida, hechos estilizados \\[0.3em]
    \key{WDI} (Banco Mundial) & 1960--hoy, 217 países & PIB, salud, educación, Gini \\[0.3em]
    \key{Maddison Project} & 1 d.C.--2022 & Historia de muy largo plazo \\[0.3em]
    \key{PWT} (Penn World Tables) & 1950--2019 & Contabilidad del crecimiento (L3) \\[0.3em]
    \key{CASEN} & Chile 1990--hoy & Micro-datos Chile, pobreza, desigualdad \\
    \bottomrule
  \end{tabular}
  \vspace{0.4em}
  \begin{keybox}
    \small Para T1, usen \textbf{OWID + WDI}. Son de acceso libre y tienen APIs en R y Python.
  \end{keybox}
\end{frame}

% ============================================================
% SLIDE 36: Cómo leer un gráfico de OWID
% ============================================================
\begin{frame}{Cómo leer un gráfico de OWID}
  \begin{columns}[T]
    \column{0.54\textwidth}
      \begin{center}
      \begin{tikzpicture}[scale=0.85,
        ann/.style={draw=accentamber, fill=accentamber!10, rounded corners=2pt,
                    font=\tiny, align=center, inner sep=3pt}
      ]
        \draw[primarynavy, thick] (0,0) rectangle (6.5,4.5);
        \fill[primarynavy!10] (0,3.8) rectangle (6.5,4.5);
        \node[font=\tiny\bfseries, primarynavy] at (3.25,4.15)
          {PIB per cápita, 1990--2022 (USD PPP 2017)};
        \draw[->] (0.5,0.3) -- (6.2,0.3) node[right]{\tiny Año};
        \draw[->] (0.5,0.3) -- (0.5,3.6) node[above]{\tiny USD PPP};
        \draw[accentamber, thick]
          plot[smooth] coordinates {(0.5,0.5)(2,1.0)(3.5,1.8)(5,2.3)(6,2.6)};
        \draw[warmteal, thick, dashed]
          plot[smooth] coordinates {(0.5,1.8)(2,2.2)(3.5,2.7)(5,3.0)(6,3.1)};
        \fill[accentamber] (1.0,3.4) rectangle (1.5,3.55);
          \node[right, font=\tiny] at (1.5,3.47) {Mi país};
        \fill[warmteal] (1.0,3.1) rectangle (1.5,3.25);
          \node[right, font=\tiny] at (1.5,3.17) {Chile};
        \node[ann, above] at (3.25,4.5) {\textbf{1} Variable y unidad};
        \node[ann, right] at (6.5,2.0) {\textbf{2} Fuente\\ al pie};
      \end{tikzpicture}
      \end{center}
    \column{0.42\textwidth}
      \textbf{Al leer cualquier gráfico de OWID, identifiquen:}
      \begin{enumerate}
        \item ¿Cuál es la \key{variable}?
        \item ¿Cuáles son las \key{unidades}? (PPP, nominal, año base)
        \item ¿Cuál es la \key{fuente subyacente}? (WDI, Maddison, PWT)
        \item ¿Cuál es el \key{período} cubierto?
      \end{enumerate}
      \vspace{0.4em}
      \muted{\small Siempre citar la fuente subyacente, no solo ``OWID''.}
  \end{columns}
\end{frame}

% ============================================================
% SLIDE 37: Taller OWID
% ============================================================
\begin{frame}{Taller: explorando datos en tiempo real}
  \begin{columns}[T]
    \column{0.56\textwidth}
      \textbf{Actividad (5 minutos):}
      \begin{enumerate}
        \item Vayan a \href{https://ourworldindata.org}{ourworldindata.org}
        \item Busquen el PIB per cápita PPP de su país
        \item Agréguenle Chile y un país OCDE
        \item Observen:
          \begin{itemize}
            \item ¿Cuándo comenzó a crecer su país?
            \item ¿Está convergiendo a Chile?
            \item ¿Hay quiebres estructurales visibles?
          \end{itemize}
      \end{enumerate}
    \column{0.40\textwidth}
      \begin{keybox}
        \textbf{Esto es exactamente T1.}\\[0.3em]
        En T1 harán lo mismo pero con 4 variables y una narrativa escrita de 2 páginas.\\[0.3em]
        \small Hoy practican; el viernes entregan.
      \end{keybox}
  \end{columns}
\end{frame}

% ============================================================
% HERRAMIENTAS C: CORRELACIÓN VS. CAUSALIDAD
% ============================================================

% ============================================================
% SLIDE 38: Scatter plots y correlación
% ============================================================
\begin{frame}{Correlación: ingresos e instituciones}
  \begin{columns}[T]
    \column{0.54\textwidth}
      \begin{center}
      \begin{tikzpicture}[scale=0.78]
        \draw[->] (0,0) -- (5.5,0) node[right]{\scriptsize Instituciones};
        \draw[->] (0,0) -- (0,4.2) node[above]{\scriptsize $\ln$ PIB pc};
        \foreach \x/\y in {
          0.3/0.5, 0.5/0.8, 0.8/1.1, 1.0/0.9, 1.2/1.4,
          1.5/1.7, 1.8/2.1, 2.1/1.9, 2.3/2.5, 2.6/2.7,
          2.9/3.0, 3.1/3.0, 3.4/3.2, 3.6/3.5, 3.9/3.6,
          4.1/3.7, 4.4/3.8, 4.7/3.9
        }{
          \fill[primarynavy, opacity=0.5] (\x,\y) circle (2.5pt);
        }
        \fill[accentamber] (3.3,3.2) circle (4pt)
          node[above, font=\scriptsize]{\textbf{Chile}};
        \draw[negativered, thick, dashed] (0.2,0.3) -- (5.0,4.0);
      \end{tikzpicture}
      \end{center}
    \column{0.42\textwidth}
      \begin{itemize}
        \item Correlación clara: mejores instituciones $\rightarrow$ mayor ingreso
        \item Pero, ¿qué causa qué?
        \item ¿Son ricas porque tienen buenas instituciones?
        \item ¿O tienen buenas instituciones porque son ricas?
      \end{itemize}
      \vspace{0.4em}
      \begin{highlightbox}
        \scriptsize \citet{AcemogluJohnsonRobinson2001_colonial}: base empírica de la Clase~6.
      \end{highlightbox}
  \end{columns}
\end{frame}

% ============================================================
% SLIDE 39: El problema de la causalidad
% ============================================================
\begin{frame}{El problema fundamental de la causalidad}
  \begin{columns}[T]
    \column{0.50\textwidth}
      \begin{definitionbox}{Problemas de identificación}
        \small
        \begin{enumerate}
          \setlength{\itemsep}{0.4em}
          \item \textbf{Causalidad inversa:} Si $X \rightarrow Y$ pero también $Y \rightarrow X$, OLS no identifica el efecto causal.\\
                \muted{Ej: Instituciones $\leftrightarrow$ Ingresos}
          \item \textbf{Variable omitida:} Si $Z$ causa tanto $X$ como $Y$, la correlación entre $X$ e $Y$ es espuria.\\
                \muted{Ej: Geografía $\rightarrow$ \{Instituciones, Ingresos\}}
        \end{enumerate}
      \end{definitionbox}
    \column{0.46\textwidth}
      \begin{center}
      \begin{tikzpicture}[
        n/.style={draw=primarynavy, circle, minimum size=0.9cm,
                  align=center, font=\small},
        arr/.style={->, thick}
      ]
        \node[n] (X) at (0,2) {$X$};
        \node[n] (Y) at (3.2,2) {$Y$};
        \node[n, fill=accentamber!25] (Z) at (1.6,4.0) {$Z$};
        \draw[arr, accentamber, <->] (X) to[bend left=15]
          node[above, font=\small]{?} (Y);
        \draw[arr, slategray] (Z) -- (X);
        \draw[arr, slategray] (Z) -- (Y);
        \node[font=\scriptsize, text=slategray] at (1.6,3.1) {confundidor};
      \end{tikzpicture}
      \end{center}
      \vspace{0.3em}
      \muted{\small Necesitamos estrategias de identificación que separen correlación de causalidad.}
  \end{columns}
\end{frame}

% ============================================================
% SLIDE 40: ¿Qué queremos saber? Pensamiento contrafactual
% ============================================================
\begin{frame}{¿Qué queremos saber? Pensamiento contrafactual}
  \begin{definitionbox}{Contrafactual}
    ¿Qué hubiera pasado con el país $i$ si hubiera tenido instituciones mejores (o peores), manteniendo todo lo demás igual?
  \end{definitionbox}
  \vspace{0.5em}
  \begin{columns}[T]
    \column{0.48\textwidth}
      \textbf{El problema fundamental:}
      \begin{itemize}
        \item Nunca observamos el contrafactual
        \item Cada país es único: no hay ``gemelo'' perfecto
        \item La historia no se repite bajo condiciones controladas
      \end{itemize}
    \column{0.48\textwidth}
      \textbf{Las soluciones que estudiaremos:}
      \begin{itemize}
        \item \key{IV}: variación exógena en $X$
        \item \key{RCT}: asignación aleatoria
        \item \key{DiD}: comparar tendencias
        \item \key{RDD}: explotar discontinuidades
      \end{itemize}
  \end{columns}
\end{frame}

% ============================================================
% SLIDE 41: Preview tres estrategias
% ============================================================
\begin{frame}{Tres estrategias de identificación: un mapa del curso}
  \small
  \begin{tabular}{@{}p{1.8cm}p{3.8cm}p{5.5cm}@{}}
    \toprule
    \textbf{Método} & \textbf{Idea central} & \textbf{Ejemplo en este curso} \\
    \midrule
    \key{OLS} &
      Controlar por observables &
      Regresión de Solow: $\ln Y_i = \alpha + \beta \ln (s/n) + \ldots$ (L3) \\[0.5em]
    \key{IV} &
      Variable instrumental exógena &
      Mortalidad de colonizadores como IV para calidad institucional (AJR, L6) \\[0.5em]
    \key{RCT} &
      Asignación aleatoria al tratamiento &
      Desparasitación en Kenia (Kremer \& Miguel, L10) \\
    \bottomrule
  \end{tabular}
  \vspace{0.5em}
  \begin{keybox}
    \small En cada clase señalaremos explícitamente qué estrategia de identificación usa el paper que analizamos y por qué.
  \end{keybox}
\end{frame}

% ============================================================
% HERRAMIENTAS D: RCTs Y EXPERIMENTOS
% ============================================================

% ============================================================
% SLIDE 42: El experimento ideal
% ============================================================
\begin{frame}{El experimento ideal: ensayo aleatorio controlado (RCT)}
  \begin{columns}[T]
    \column{0.52\textwidth}
      \begin{center}
      \begin{tikzpicture}[
        box/.style={draw=primarynavy, fill=lightgray, rounded corners=3pt,
                    minimum width=2.3cm, minimum height=0.9cm, align=center,
                    font=\small},
        arr/.style={->, thick, primarynavy}
      ]
        \node[box] (pop) at (0,3.2) {Población};
        \node[box, fill=positivegreen!20] (T) at (-1.6,1.6) {Tratados};
        \node[box, fill=negativered!10] (C) at (1.6,1.6) {Control};
        \node[box, fill=positivegreen!20] (YT) at (-1.6,0) {$Y_T$};
        \node[box, fill=negativered!10] (YC) at (1.6,0) {$Y_C$};
        \draw[arr] (pop.south west) -- (T.north);
        \draw[arr] (pop.south east) -- (C.north);
        \draw[arr] (T.south) -- (YT.north);
        \draw[arr] (C.south) -- (YC.north);
        \node[text=accentamber, font=\small\bfseries] at (0,-0.8)
          {\ensuremath{\hat{\tau} = \bar{Y}_{T} - \bar{Y}_{C}}};
        \node[font=\scriptsize, text=warmteal] at (0,2.4) {Aleatorio};
      \end{tikzpicture}
      \end{center}
    \column{0.44\textwidth}
      \textbf{¿Por qué funciona?}
      \begin{itemize}
        \item Con asignación aleatoria:
          \[
            E[Y^{(0)} | T=1] = E[Y^{(0)} | T=0]
          \]
        \item El grupo control es el contrafactual perfecto
        \item No hay selección, no hay confundidores
      \end{itemize}
      \vspace{0.3em}
      \begin{keybox}
        \small El RCT es el \textbf{``gold standard''} para evaluar políticas de desarrollo.
      \end{keybox}
  \end{columns}
\end{frame}

% ============================================================
% SLIDE 43: RCTs en economía del desarrollo
% ============================================================
\begin{frame}{RCTs en economía del desarrollo: la revolución J-PAL}
  \begin{columns}[T]
    \column{0.56\textwidth}
      \textbf{Ejemplos emblemáticos (Clase 10 en detalle):}
      \begin{itemize}
        \item \textbf{Desparasitación en Kenia} \citet{KremerMiguel2004_worms}:\\
              Tratar gusanos intestinales $\rightarrow$ \positive{+25\% asistencia escolar}
        \vspace{0.3em}
        \item \textbf{Construcción de escuelas en Indonesia} \citet{Duflo2001_schooling}:\\
              61{,}000 escuelas nuevas (1973--78) $\rightarrow$ \positive{+0.1--0.2 años educación}
        \vspace{0.3em}
        \item \textbf{Microcrédito en India} \citet{BanerjeeEtAl2015_microfinance}:\\
              Acceso a crédito $\rightarrow$ \muted{efectos mixtos, no la ``bala de plata''}
      \end{itemize}
    \column{0.40\textwidth}
      \begin{resultbox}
        \textbf{Premio Nobel 2019:}\\[0.3em]
        Abhijit Banerjee, Esther Duflo y Michael Kremer por su contribución a la economía del desarrollo experimental.
      \end{resultbox}
      \vspace{0.4em}
      \muted{\small El J-PAL (MIT) ha evaluado más de 1{,}000 RCTs en 93 países.}
  \end{columns}
\end{frame}

% ============================================================
% SLIDE 44: Limitaciones de los RCTs
% ============================================================
\begin{frame}{Limitaciones de los RCTs: no son una panacea}
  \begin{columns}[T]
    \column{0.55\textwidth}
      \begin{highlightbox}
        \textbf{Tres limitaciones clave:}
        \begin{enumerate}
          \setlength{\itemsep}{0.3em}
          \item \textbf{Escala:} evalúan intervenciones locales; no capturan efectos de equilibrio general.
          \item \textbf{Validez externa:} un RCT en Kenia no se generaliza automáticamente a Chile.
          \item \textbf{Ética:} ¿es válido negar tratamiento al grupo control? ¿quién decide qué se experimenta?
        \end{enumerate}
      \end{highlightbox}
    \column{0.41\textwidth}
      \vspace{0.5em}
      \begin{keybox}
        \small Los RCTs son herramientas poderosas, pero la economía del desarrollo necesita \emph{todas} sus herramientas.
      \end{keybox}
      \vspace{0.5em}
      \muted{\small El método que usamos depende de la pregunta que hacemos.}
  \end{columns}
\end{frame}

% ============================================================
% SLIDE 45: Hilo conductor metodológico
% ============================================================
\begin{frame}{Hilo conductor: el método sigue la pregunta}
  \begin{center}
  \small
  \begin{tabular}{@{}p{3.8cm}p{2.5cm}p{4.5cm}@{}}
    \toprule
    \textbf{Pregunta de investigación} & \textbf{Método} & \textbf{Clase} \\
    \midrule
    ¿Por qué crecen los países?             & OLS/Calibración & L3: Solow \\[0.2em]
    ¿Instituciones causan riqueza?          & IV              & L6: Instituciones \\[0.2em]
    ¿Funciona la desparasitación?           & RCT             & L10: Salud \\[0.2em]
    ¿El microcrédito reduce pobreza?        & RCT/IV          & L11: Crédito \\
    \bottomrule
  \end{tabular}
  \end{center}
  \vspace{0.5em}
  \begin{keybox}
    \textbf{Regla de este curso:} Siempre preguntamos \emph{¿cómo lo sabemos?} para cada resultado empírico. El método importa tanto como el resultado.
  \end{keybox}
\end{frame}

% ============================================================
% BLOQUE 4: CIERRE
% ============================================================
\begin{frame}[plain]
  \begin{center}
    \vspace{1.5em}
    {\Large \color{primarynavy}\textbf{Cierre}}\\[0.6em]
    {\large \color{warmteal}\textit{Resumen y próximos pasos}}
  \end{center}
\end{frame}

% ============================================================
% SLIDE 46: ¿Qué aprendimos hoy?
% ============================================================
\begin{frame}{¿Qué aprendimos hoy?}
  \begin{keybox}
    \begin{enumerate}
      \item \textbf{Los tres misterios del desarrollo:} crecimiento, desigualdad y estancamiento. El curso los aborda sistemáticamente en 12 clases.
      \vspace{0.2em}
      \item \textbf{Los hechos estilizados:} brecha 80:1, divergencia histórica, correlación ingreso-salud-educación, Chile en perspectiva.
      \vspace{0.2em}
      \item \textbf{La evaluación:} tareas (35\%), controles (25\%), examen (30\%), participación (10\%). Country Portfolio comienza hoy.
      \vspace{0.2em}
      \item \textbf{Herramientas básicas:} logaritmos, regla del 70, crecimiento compuesto, datos PPP vs.\ nominal.
      \vspace{0.2em}
      \item \textbf{Método:} correlación $\neq$ causalidad. Usaremos OLS, IV y RCTs a lo largo del semestre.
    \end{enumerate}
  \end{keybox}
\end{frame}

% ============================================================
% SLIDE 47: Lectura para la próxima clase
% ============================================================
\begin{frame}{Lectura para la próxima clase}
  \begin{columns}[T]
    \column{0.50\textwidth}
      \textbf{Obligatoria:}
      \begin{itemize}
        \item \citet{AshrafWeil2025_growth}, capítulos 1--2\\
              \muted{\small El modelo maltusiano y la transición demográfica}
        \vspace{0.3em}
        \item \citet{Pritchett1997_divergence}\\
              \muted{\small ``Divergence, Big Time'' --- 8 páginas, altamente accesible}
      \end{itemize}
    \column{0.46\textwidth}
      \textbf{Recomendada:}
      \begin{itemize}
        \item \citet{Deaton2013_escape}, Introducción\\
              \muted{\small Las primeras 30 páginas cuentan la historia global del escape de la pobreza}
      \end{itemize}
      \vspace{0.5em}
      \begin{highlightbox}
        \small \textbf{Pregunta guía para L2:}\\
        ¿Qué mecanismo impidió el crecimiento per cápita durante milenios y por qué se rompió en el siglo XVIII?
      \end{highlightbox}
  \end{columns}
\end{frame}

% ============================================================
% SLIDE 48: Tarea 1 recordatorio
% ============================================================
\begin{frame}{Tarea 1: próximos pasos}
  \begin{columns}[T]
    \column{0.54\textwidth}
      \begin{keybox}
        \textbf{T1 -- Perfil de país}\\[0.3em]
        \begin{itemize}
          \item \textbf{Plazo:} \underline{[fecha --- 7 días desde hoy]}
          \item \textbf{Entrega:} PDF en SIDING
          \item \textbf{Extensión:} 2 páginas de texto + figuras
          \item \textbf{Formato:} libre (Word, LaTeX, R Markdown)
        \end{itemize}
      \end{keybox}
    \column{0.42\textwidth}
      \textbf{Para hoy:}
      \begin{enumerate}
        \item Registrar su país en SIDING
        \item Explorar OWID con su país
        \item Descargar datos de WDI
        \item Empezar las 4 figuras
      \end{enumerate}
      \vspace{0.3em}
      \muted{\small Pueden usar IA para el código; deben declarar el uso y mostrar su propio razonamiento económico.}
  \end{columns}
\end{frame}

% ============================================================
% SLIDE 49: Próxima clase
% ============================================================
\begin{frame}{Próxima clase: L2 -- Malthus y la transición demográfica}
  \begin{columns}[T]
    \column{0.52\textwidth}
      \begin{methodbox}
        \textbf{La trampa que duró 10{,}000 años:}\\[0.4em]
        \begin{itemize}
          \item ¿Por qué el ingreso per cápita no crecía antes de 1750?
          \item El modelo maltusiano: población + recursos fijos
          \item ¿Qué rompió la trampa? La Revolución Industrial
          \item La transición demográfica: de muchos hijos pobres a pocos hijos ricos
        \end{itemize}
      \end{methodbox}
    \column{0.44\textwidth}
      \begin{highlightbox}
        \small En 1800, el habitante promedio de Europa vivía cerca del nivel de subsistencia, apenas por encima del año 1500. ¿Cómo es posible? Malthus tenía la respuesta---y era aterradora.
      \end{highlightbox}
      \vspace{0.4em}
      \muted{\small Textbook: \citet{AshrafWeil2025_growth}, capítulos 3--4}
  \end{columns}
\end{frame}

% ============================================================
% SLIDE 50: Referencias
% ============================================================
\begin{frame}[allowframebreaks]{Referencias}
  \small
  \bibliography{../Bibliography_base}
\end{frame}

\end{document}
